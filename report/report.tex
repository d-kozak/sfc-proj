\documentclass[12pt,a4paper,titlepage]{article}
\usepackage[left=2.5cm,text={16cm,20cm},top=4cm]{geometry}
\usepackage[T1]{fontenc}
\usepackage[czech]{babel}
\usepackage[utf8]{inputenc}
% dalsi balicky
\usepackage{graphicx}
\usepackage{enumitem}
\usepackage{indentfirst}
\usepackage{float}
\usepackage{svg}
\usepackage{amsmath}
\usepackage{url}
\usepackage{graphics}
\usepackage{graphicx}
\usepackage{multicol}
\graphicspath{ {images/} }
\usepackage[bookmarksopen,colorlinks,plainpages=false,urlcolor=blue,
unicode,linkcolor=black]{hyperref}

\bibliographystyle{czplain}

%úvodzovky
\providecommand{\uv}[1]{\quotedblbase #1\textquotedblleft}

\begin{document}

\begin{titlepage}
\begin{center}
    {
    	\Huge\textsc{Vysoké učení technické v~Brně}}\\
    \smallskip
    {
    	\huge\textsc{Fakulta informačních technologií}}\\
    \bigskip
    \vspace{\stretch{0.382}} %pomery odpovedajúcí zlatému rezu    
    \huge{Soft computing}\\
    \smallskip
    \Huge{Projekt - fuzzy inference}\\
    \vspace{\stretch{0.618}}
\end{center}
    {\Large \today \hfill David Kozák (xkozak15)  }\\
\end{titlepage}

\newpage
\tableofcontents
\newpage

\section{Úvod}
Tento text slouží jako dokumentace projektu do předmětu Soft Computing na téma Fuzzy inference. Byla vypracována v rámci akademického roku 2017/2018. 

\section{Fuzzy logika}
Jedná se o rozšíření klasické booleovské logiky umožňující vyjádřit nepřesnost či neurčitost. Příslušnost prvku do fuzzy množiny je vyjádřena reálným číslem z intervalu <0,1>. 

\section{Uživalská příručka}
Tato sekce se zabývá dvěma tématy, implementačními detaily a příkazy pro překlad a spuštění projektu.

\subsection{Implementační detaily}
Projekt byl implementován v programovacím jazyku Java s využitím GUI frameworku JavaFX. Krom toho jsou též využity externí knihovny ProjectLombok, junit, mockito, afterburner.fx a aquafx.

\subsection{Překlad a spuštění}
Projekt lze přeložit příkazem ant. Jelikož build file stahuje externí knihovny, je důležité mít při překladu přístup k internetu. Výstupem překladu je jar file \textit{sfc-proj-app.jar} ve složce \textit{target}. Spustit ho tedy můžete příkazem \textit{java -jar target/sfc-proj-app.jar}.

Jelikož původní projekt byl vytvořen jako maven projekt, je ho též možno přeložit příkazem \textit{mvn clean install}, tedy za přepokladu, že je maven na daném stroji nainstalovaný.


\section*{Reference}
\begin{enumerate}[label={[\arabic*]}]
\item PERINGER P. Slajdy k přednáškám modelování a simulace, 2016. Verze  2016-09-20 [cit. 2016-12-05][Online] \\
     \href{https://www.fit.vutbr.cz/study/courses/IMS/public/prednasky/IMS.pdf}
          {https://www.fit.vutbr.cz/study/courses/IMS/public/prednasky/IMS.pdf}
     \label{prezentace}

\item PERINGER P. SIMulation LIBrary for C++, 2011. [Online] \\
    \href{https://www.fit.vutbr.cz/~peringer/SIMLIB/}
         {https://www.fit.vutbr.cz/~peringer/SIMLIB/}
    \label{demo1}

\item HRUBÝ M. Demonstrační cvičení IMS \#1, [cit. 2016-12-05][Online] \\
    \href{http://perchta.fit.vutbr.cz:8000/vyuka-ims/uploads/1/ims-demo1.pdf}
        {http://perchta.fit.vutbr.cz:8000/vyuka-ims/uploads/1/ims-demo1.pdf}
    \label{demo2}

\item HRUBÝ M. IMS democvičení \#2, [cit. 2016-12-05][Online] \\
    \href{http://perchta.fit.vutbr.cz:8000/vyuka-ims/uploads/1/diskr2-2011.pdf}
        {http://perchta.fit.vutbr.cz:8000/vyuka-ims/uploads/1/diskr2-2011.pdf}
    \label{simlib}

\item Statistické informace Googlu o modelované prodejně \\
     \href{http://goo.gl/6zOMvi}
          {http://goo.gl/6zOMvi}
     \label{google-shop}
\end{enumerate}
\end{document}